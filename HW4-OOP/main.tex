\documentclass[a4paper,12pt]{article}
\usepackage{url}
\usepackage{listings}
\usepackage{color}
\usepackage{hyperref}
\usepackage{graphicx}
\usepackage{booktabs}
%\usepackage{mdframed}
\usepackage{adjustbox}

\definecolor{dkgreen}{rgb}{0,0.6,0}
\definecolor{gray}{rgb}{0.5,0.5,0.5}
\definecolor{mauve}{rgb}{0.58,0,0.82}

\lstset{frame=tb,
	language=C++,
	aboveskip=3mm,
	belowskip=3mm,
	showstringspaces=false,
	columns=flexible,
	basicstyle={\small\ttfamily},
	numbers=none,
	numberstyle=\tiny\color{gray},
	keywordstyle=\color{blue},
	commentstyle=\color{dkgreen},
	stringstyle=\color{mauve},
	breaklines=true,
	breakatwhitespace=true,
	tabsize=3
}
\begin{document}
	
	\title{CS-224 Object Oriented Programming and Design Methodologies }
	\author{Homework 04}
	\date{Spring 2022}
	\maketitle
	\section{Guidelines}
	
	You need to submit this homework on  {\color{blue}18th March at 8pm}, on LMS. Late submissions are allowed until {\color{red} 20th March 11:59pm}, which will be penalized by 20\%. Your work will not be accepted once the submission is closed on LMS.

	
	\begin{itemize}
		\item You need to do this assignment in a group of two students.
		\item You will submit your assignment to LMS (only one member of the group will submit).
		\item Clearly mention the group composition in submitted file name e.g. AhmadHassan\_ah01345\_BatoolAiman\_ba03451.zip. 
		\item You need to follow the best programming practices 
		\item Submit assignment on time; late submissions will not be accepted.
		\item Some assignments will require you to submit multiple files. Always Zip and send them.
		\item It is better to submit incomplete assignment than none at all.
		\item It is better to submit the work that you have done yourself than what you have plagiarized.
		\item It is strongly advised that you start working on the assignment the day you get it. Assignments WILL take time.
%		\item Every assignment you submit should be a single zipped file containing all the other files. Suppose your name is John Doe and your id is 0022 so the name of the submitted file should be JohnDoe0022.zip
		\item DO NOT send your assignment to your instructor, if you do, your assignment will get ZERO for not following clear instructions.
		\item You can be called in for Viva for any assignment that you submit
	\end{itemize}
	

	

	\section{HUMania}
	
	A sample code is given in HUMania folder, if you run it you can see a pigeon is drawn. This example creates just one object of Pigeon to show how things are drawn in SDL. Refer to \texttt{Pigeon.hpp/cpp and HUMania.cpp $ \Rightarrow $ drawObjects()}.
	
	 You are required to:
	 \begin{itemize}
	 	\item Create a \texttt{Pigeon} class (see the pigeon.hpp/cpp), that will contain attributes and functions (\texttt{fly, draw}) related to a pigeon. The \texttt{fly} function flies the pigeon gradually to right side, and rotates through the screen. 
	 	
	 	\item Create a \texttt{Butterfly} class (create butterfly.cpp/hpp files), that will contain attributes and functions (\texttt{fly, draw}) related to butterfly. The fly function should take the butterfly right-down direction. Once a butterfly reaches to bottom of the screen, it starts flying right-up direction. Once it reaches top of the screen it moves right-down. Similar to the pigeon, it should rotate through the screen. 
	 	
	 	\item Create a \texttt{Bee} class (Create bee.cpp/hpp files), that will contain attributes and functions (\texttt{fly, draw}) related to a bee. The fly function should make it fly towards right only. During fly it should hover (doesn't move forward) for a while over a random interval. You may choose 1\% probability in every frame to decide whether it starts hover, and it keeps hovering for 10 frames. As a bee reaches to right most border of screen, it exits from the game, hence the object must be removed from the bees vector.  
	 	
	 	\item Every object animates three of the images provided in the assets file. The draw function is only drawing the object.
	 	
	 	\item As you click on the screen, one of the above objects is created randomly. You'll maintain three vectors (pigeons, butterflies, bees) in \texttt{HUMania.hpp/cpp} to store objects of different classes. The object that you create on the click will be pushed into corresponding vector. Refer to \texttt{HUMania.cpp $ \Rightarrow $ createObject()}, where you get mouse coordinates.
	 	
	 	\item You have to create objects dynamically with \texttt{new} operator, hence the vectors should hold pointers to all of the objects. Remember to delete the objects when game is ended, and when the bee objects are removed from vector.
	 	
	 	\item Finally, you iterate over all the elements of vectors, and call their fly and draw functions.
	 	
	 	\item Please refer to \path{Solution.exe} file to see it all in action.
	 	
	 	\item Are you having fun?? You are more than welcome to add more stuff to make this game interesting, e.g. some natural random movement of butterflies, sitting them on ground, pigeons sweeping etc. [\textit{It doesn't carry any marks}]
	 \end{itemize}
 

	\begin{figure}
		\includegraphics[width=\linewidth]{sdlDrawing}
		\caption{SDL Drawing Basics}
		\label{fig:sdlDrawing}
	\end{figure}  

	
	
	\subsection{SDL Drawing Basics}
	
	The basic drawing function in SDL is very simple, you need two SDL\_Rect variables to draw a portion of image from assets file to the canvas. \texttt{SDL\_Rect} is a simple structure containing \texttt{\{x, y, w, h\} }attributes. \texttt{(x, y)} is the top-left corner, and \texttt{w, h} are width and height of rectangle. You define a \texttt{srcRect} for desired object in assets file, and define a \texttt{moverRect} for this image to be drawn on desired location on canvas. Refer to Figure \ref{fig:sdlDrawing} for all this process.  Finally you call 
	
	\noindent \texttt{SDL\_RenderCopy(gRenderer, assets, \&pigeonSrc, \&pigeonMover);}
	
	\noindent that displays this image to the canvas, voila!!!. Refer to \texttt{assets.png} file for all the required image assets.
	
	You can draw as many objects in the \texttt{HUMania.cpp $ \Rightarrow $ drawObjects()}, as you want. Since this function is called infinitely, you can change the \texttt{x, y} attributes of \texttt{moverRect} to move the objects on screen, and you can change the \texttt{srcRect} values to get a flying animation.
	


\section{\texttt{std::vector} Tutorial} \label{vectorTutorial}

Following is a basic example to work with vector. Complete reference for C++ vector is given here \url{https://en.cppreference.com/w/cpp/container/vector}
\begin{lstlisting}
#include<iostream>
#include<vector>

using namespace std;

class Distance{
	int feet, inches;
	public:
	Distance(int ft, int inch): feet(ft), inches(inch){}
	void show(){
		cout<<feet<<"'"<<inches<<"\""<<endl;
	}
};

int main(){
	vector<Distance*> dst; // It's a vector that can store Distance type objects
	dst.push_back(new Distance(3, 4)); // create an object, and push it in vector
	dst.push_back(new Distance(5, 2));
	dst.push_back(new Distance(2, 7));
	dst.push_back(new Distance(7, 8));
	dst.push_back(new Distance(13, 1));
	
	for(int i=0;i<dst.size();i++)
		dst[i]->show(); // call show method of dst[i] object
		
	// deleting the objects, need to delete every single object created dynamically
	for(int i=0;i<dst.size(); i++)
		delete dst[i];
	
	dst.clear(); //clears all the items from vector
	
}

//////////////// Output: ///////////////////
3'4"
5'2"
2'7"
7'8"
13'1"
\end{lstlisting}


\section{Some important points:} 

\begin{itemize}
	\item Sample code is there for your benefit. If you are going to use it, understand how it works. 
	\item You do not need to follow the code given exactly. You can make changes where you see fit provided that it makes sense.
	\item Make the class declarations in hpp files, and provide function implementations in cpp files. Don't use hpp files for implementation purposes.
	\item \href{https://www.techiedelight.com/remove-elements-vector-inside-loop-cpp/}{A tutorial given here} to remove the elements from vector, you might need it to remove bees as they exit the screen.
	%		\item Implement Q1 prior to implementing Q2, it will help you to implement linked list.
	%		\item Where necessary, declare your own functions inside classes. Make sure why you would keep a function as private or public.
	\item As a general rule, class's deta is private, and functions are public. Don't use getter/setter functions to manipulate data, rather think in object oriented directions and provide all the functionality in the class.
	\item Complete reference for C++ vector is given here \url{https://en.cppreference.com/w/cpp/container/vector}
	\item You need to define separate \path{*.hpp} and \path{*.cpp} files for all the classes.
	\item Exact x,y,w,h values for images in assets file can be found by \url{http://www.spritecow.com/}. 
	\item A tutorial for file I/O is given \url{http://www.cplusplus.com/doc/tutorial/files/}. 
	\item You should take \url{www.cplusplus.com} and \url{www.cppreference.com} as primary web source to search about C++
	\item You have to follow best OOP practices as discussed in lectures.
\end{itemize}

\section{How to compile}
Open the given \texttt{Seeplusia} folder in vscode by choosing \texttt{File $\Rightarrow$ Open Folder}. The game can be run by simply pressing F5 from vscode. If due to some reason it doesn't work, then go compiling and running it from terminal, as explained in \texttt{how to compile.txt}


\section{Rubric}
\begin{table}[!h]
	\centering
	\begin{tabular}{llc}
		\toprule
%		Warnings/Errors	& The code had no warnings/errors/	& 1 \\
%		Comments &	The code was properly commented	& 1 \\
		Coding	& The code followed best practices guideline &	1 \\
		OOP Concepts & The code followed best OOP practices & 2 \\
		 Memory &	Dynamic memory management is done properly	& 2\\
		Functionality	& All the functionality is implemented as described above	& 5 \\
		\midrule
		Total & & 10\\
		\bottomrule
	\end{tabular}
	\caption{Grading Rubric}
	\label{Grading}
\end{table}


	
	
\end{document}